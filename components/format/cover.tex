\section{表紙}

表紙は,{\tt $\backslash$maketitle} によって作成するため,以下の項目に相
当する文字列をそれぞれ記述する.

% textlint-disable
\begin{description} \parskip=1pt
    \item{題目: }
          題目は{\tt $\backslash$title} に記述する.行替えを行う場合は$\backslash$
          $\backslash$ を入力する.ただし,題目の最後に$\backslash$
          $\backslash$ を入力するとコンパイルが通らなくなるので注意する.
          なお,4行以上の題目の場合,表紙ページがあふれるためスタイルファ
          イル``sie-euc.sty''を変更する必要がある.
    \item{著者名: }
          著者名は{\tt $\backslash$author} に記述する.
    \item{学位: }
          学位名は{\tt $\backslash$degree} に記述する.
    \item{指導教員名: }
          指導教教員は{\tt $\backslash$advisor} に記述する.
    \item{専攻名: }
          専攻名は{\tt $\backslash$majorfield} に記述する.
    \item{学位プログラム名: }
          学位プログラム名は{\tt $\backslash$programfield} に記述する.
    \item{年月: }
          年月は{\tt $\backslash$yearandmonth} に記述する.
\end{description}
% textlint-enable